\chapter*{Заключение}                       % Заголовок
\addcontentsline{toc}{chapter}{Заключение}  % Добавляем его в оглавление

%% Согласно ГОСТ Р 7.0.11-2011:
%% 5.3.3 В заключении диссертации излагают итоги выполненного исследования, рекомендации, перспективы дальнейшей разработки темы.
%% 9.2.3 В заключении автореферата диссертации излагают итоги данного исследования, рекомендации и перспективы дальнейшей разработки темы.
%% Поэтому имеет смысл сделать эту часть общей и загрузить из одного файла в автореферат и в диссертацию:

Основные результаты работы заключаются в следующем.
%% Согласно ГОСТ Р 7.0.11-2011:
%% 5.3.3 В заключении диссертации излагают итоги выполненного исследования, рекомендации, перспективы дальнейшей разработки темы.
%% 9.2.3 В заключении автореферата диссертации излагают итоги данного исследования, рекомендации и перспективы дальнейшей разработки темы.
\begin{enumerate}
  \item На основе анализа \ldots
  \item Численные исследования показали, что \ldots
  \item Математическое моделирование показало \ldots
  \item Для выполнения поставленных задач был создан \ldots
\end{enumerate}

Можно надеяться, что значения параметров электрических дипольных переходов окажутся полезными при анализе явления невзаимности в \cbo\ на других частотах. Кроме того, развитая теория магнитоэлектрической связи может быть использована при описании других магнетиков, у которых отсутствует центр инверсии.

Основной материал диссертации составляют статьи \cite{Eremin2021, Nurmukhametov2022}. Также имеются публикации по темам, связанным с темой работы: изучение возбужденного состояния 1.4 эВ \cbo\ \cite{Kopteva2022}; изучение магнитоэлектрических эффектов в другом магнетике --- \ce{FeZnMo3O8} \cite{Vasin2022}.

В заключение автор выражает благодарность и большую признательность научному руководителю
Еремину М. В. за научное руководство, обучение и поддержку.
