\chapter*{Заключение}                       % Заголовок
\addcontentsline{toc}{chapter}{Заключение}  % Добавляем его в оглавление

%% Согласно ГОСТ Р 7.0.11-2011:
%% 5.3.3 В заключении диссертации излагают итоги выполненного исследования, рекомендации, перспективы дальнейшей разработки темы.
%% 9.2.3 В заключении автореферата диссертации излагают итоги данного исследования, рекомендации и перспективы дальнейшей разработки темы.
%% Поэтому имеет смысл сделать эту часть общей и загрузить из одного файла в автореферат и в диссертацию:

Основные результаты работы заключаются в следующем.
%% Согласно ГОСТ Р 7.0.11-2011:
%% 5.3.3 В заключении диссертации излагают итоги выполненного исследования, рекомендации, перспективы дальнейшей разработки темы.
%% 9.2.3 В заключении автореферата диссертации излагают итоги данного исследования, рекомендации и перспективы дальнейшей разработки темы.
\begin{enumerate}
  \item На основе анализа \ldots
  \item Численные исследования показали, что \ldots
  \item Математическое моделирование показало \ldots
  \item Для выполнения поставленных задач был создан \ldots
\end{enumerate}

Можно надеяться, что значения параметров электрических дипольных переходов окажутся полезными при анализе явления невзаимности в \cbo\ на других частотах. Кроме того, развитая теория магнитоэлектрической связи может быть использована при описании других магнетиков, у которых отсутствует центр инверсии.

Последний параграф может включать благодарности.  В заключение автор
выражает благодарность и большую признательность научному руководителю
Иванову~И.\,И. за поддержку, помощь, обсуждение результатов и~научное
руководство. Также автор благодарит Сидорова~А.\,А. и~Петрова~Б.\,Б.
за помощь в~работе с~образцами, Рабиновича~В.\,В. за предоставленные
образцы и~обсуждение результатов, Занудятину~Г.\,Г. и авторов шаблона
*Russian-Phd-LaTeX-Dissertation-Template* за~помощь в оформлении
диссертации. Автор также благодарит много разных людей
и~всех, кто сделал настоящую работу автора возможной.
