\chapter{Динамические магнитоэлектрические эффекты}\label{ch:ch3}

В данной главе анализируются спектры поглощения и диаграммы фотолюминесценции работ \cite{Toyoda2015, Toyoda2016} (см. \cref{sec:ch1/sec3}), а именно - переход на первое возбужденное состояние конфигурации \(3d^9\), отвечающее энергии 1.4 эВ (первый синий пик на \cref{fig:dd_transitions}). Проверяется справедливость гипотезы авторов этих работ о том, что наблюдаемые эффекты можно объяснить в рамках интерференции электрических и магнитных дипольных переходов.

\section{Таблица обыкновенная}\label{sec:ch3/sect1}



\clearpage
