\chapter{Вёрстка таблиц}\label{ch:ch3}

\section{Таблица обыкновенная}\label{sec:ch3/sect1}

Так размещается таблица:

\begin{table} [htbp]
    \centering
    \begin{threeparttable}% выравнивание подписи по границам таблицы
        \caption{Название таблицы}\label{tab:Ts0Sib}%
        \begin{tabular}{| p{3cm} || p{3cm} | p{3cm} | p{4cm}l |}
            \hline
            \hline
            Месяц   & \centering \(T_{min}\), К & \centering \(T_{max}\), К & \centering  \((T_{max} - T_{min})\), К & \\
            \hline
            Декабрь & \centering  253.575       & \centering  257.778       & \centering      4.203                  & \\
            Январь  & \centering  262.431       & \centering  263.214       & \centering      0.783                  & \\
            Февраль & \centering  261.184       & \centering  260.381       & \centering     \(-\)0.803              & \\
            \hline
            \hline
        \end{tabular}
    \end{threeparttable}
\end{table}

\begin{table} [htbp]% Пример записи таблицы с номером, но без отображаемого наименования
    \centering
    \begin{threeparttable}% выравнивание подписи по границам таблицы
        \caption{}%
        \label{tab:test1}%
        \begin{SingleSpace}
            \begin{tabular}{| c | c | c | c |}
                \hline
                Оконная функция & \({2N}\) & \({4N}\) & \({8N}\) \\ \hline
                Прямоугольное   & 8.72     & 8.77     & 8.77     \\ \hline
                Ханна           & 7.96     & 7.93     & 7.93     \\ \hline
                Хэмминга        & 8.72     & 8.77     & 8.77     \\ \hline
                Блэкмана        & 8.72     & 8.77     & 8.77     \\ \hline
            \end{tabular}%
        \end{SingleSpace}
    \end{threeparttable}
\end{table}


Некоторый текст.

\clearpage
