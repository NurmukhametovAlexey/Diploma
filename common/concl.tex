%% Согласно ГОСТ Р 7.0.11-2011:
%% 5.3.3 В заключении диссертации излагают итоги выполненного исследования, рекомендации, перспективы дальнейшей разработки темы.
%% 9.2.3 В заключении автореферата диссертации излагают итоги данного исследования, рекомендации и перспективы дальнейшей разработки темы.
\begin{enumerate}
  \item Предложен механизм магнитоэлектрической связи спинов \niIon\ и \cu\ c внешним электрическим полем в \ncbo, обусловленный совместным действием нечётного кристаллического поля и спин-орбитального взаимодействия. Оценённое значение электрической поляризации по порядку величины соответствует имеющимся экспериментальным данным. 
  \item Рассчитано изменение коэффициента поглощения света пластинкой \cbo\ во внешнем постоянном магнитном поле, соответствующее имеющимся экспериментальным данным \cite{Toyoda2015}. Наши микроскопические и теоретико-групповые расчеты поддерживают идею о том, что явление невзаимности в \cbo\ может быть объяснено интерференцией магнитных и электрических дипольных переходов.
  \item Построены диаграммы асимметрии фотолюминесценции, качественно воспроизводящие результаты работы \cite{Toyoda2016}. Расчет позволяет предсказать относительную интенсивность фотолюминесценции при произвольных направлениях волнового вектора излучения относительно кристаллографических осей \cbo, а также при различных поляризациях световой волны.

\end{enumerate}
