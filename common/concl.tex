%% Согласно ГОСТ Р 7.0.11-2011:
%% 5.3.3 В заключении диссертации излагают итоги выполненного исследования, рекомендации, перспективы дальнейшей разработки темы.
%% 9.2.3 В заключении автореферата диссертации излагают итоги данного исследования, рекомендации и перспективы дальнейшей разработки темы.
\begin{enumerate}
  \item Предложен механизм магнето-электрической связи спинов \niIon\ c внешним электрическим полем в \ncbo, обусловленный совместным действием нечётного кристаллического поля и спин-орбитального взаимодействия. Показано, что упорядочение спинов никеля в плоскости ab кристалла при антиферромагнитном упорядочении или под действием внешнего магнитного поля индуцирует электрическую поляризацию вдоль оси \textit{c} кристалла. Оценённое значение электрической поляризации по порядку величины соответствует имеющимся экспериментальным данным. Оценена связь спинов ионов меди в \cbo\ с электрическим и магнитными полями.
  \item Рассчитано изменение коэффициента поглощения света пластинкой \cbo\ во внешнем постоянном магнитном поле, соответствующее имеющимся экспериментальным данным \cite{Toyoda2015}. Наши микроскопические и теоретико-групповые расчеты поддерживают идею о том, что явление невзаимности в \cbo\ может быть объяснено интерференцией магнитных и электрических дипольных переходов.
  \item Построены диаграммы асимметрии фотолюминесценции, качественно воспроизводящие результаты работы \cite{Toyoda2016}. Развитая микроскопическая теория необратимости  позволяет рассчитать относительную интенсивность фотолюминесценции при различных направлениях волнового вектора излучения относительно кристаллографических осей \cbo, а также при различных поляризациях световой волны. В этом плане интересно продолжить исследование фотолюминесценции с б\'{о}льшим разрешением спектральных линий.

\end{enumerate}
