{\actuality} Магнитоэлектрические эффекты обусловлены наличием в термодинамическом потенциале членов, линейных как по электрическому, так и по магнитному полю \autocite{Landau}. Данные эффекты наблюдаются в магнитоэлектриках и мультиферроиках в случае, когда одновременно нарушается симметрия относительно обращения знака времени (следствие магнитного упорядочения) и симметрия относительно пространственной инверсии (следствие особенностей кристаллической структурой) - магнитными свойствами этих материалов можно управлять прикладывая внешнее электрическое поле и наоборот. Так, авторы работы \autocite{Saito2008ape} наблюдали в антиферромагнетике \cbo\ вращение намагниченности, индуцированное электрическим полем. Позднее другая группа смогла управлять электрической поляризацией \ncbo\ с помощью внешнего магнитного поля \autocite{Khanh2013}. Кроме описанных \emph{статических} магнитоэлектрических эффектов существуют также \emph{динамические} - связанные с взаимодействием с периодическими электромагнитными полями. В частности, к таковым относятся явление невзаимности (nonreciprocity) в спектрах поглощения \autocite{Toyoda2015} и явление пространственной асимметрии люминесценции (directional asymmetry of luminescence) \autocite{Toyoda2016}.  

Интерес к этим эффектам в последние десятилетия существенно возрос, в связи с реальными перспективами практических применений. На основе магнитоэлектрических материалов можно создавать магнитные запоминающие устройства с оптическим считыванием информации о доменной структуре, управляемые магнитным полем оптические диоды и пр. (см. рисунок \cref{fig:applications}) Также, разумеется, изучение данных эффектов важно для фундаментальных научных исследований.

\begin{figure}[ht]
	\centerfloat{
		\includegraphics[scale=0.4]{applications}
	}
	\caption{(a), (b) Фотографии магнитных доменов из \autocite{Toyoda2016}, полученные с помощью фотолюминесценции. (c), (d) Фотографии света, прошедшего через пластинку \cbo\, к которой приложили внешнее магнитное поле в -300 и 300~Э~\autocite{Saito2008jpsj}. (a, b) и (c, d) сделаны при одной и той же яркости и контрасте.}\label{fig:applications}
\end{figure}

Можно выделить два подхода в развитии теории магнитоэлектрических эффектов: феноменологический и микроскопический. Феноменологический подход сформулирован в работе Дзялошинского \autocite{Dzyaloshinskii1959}. Этот подход успешно использовался для анализа ряда материалов. Его применение описано во многих обзорах и оригинальных статьях \autocite{Zvezdin2008, Pyatakov2012, Popkov2016}. Микроскопический подход развит слабее и предполагает предварительный анализ энергетических схем уровней и взаимодействий магнитных ионов с обменными и электрическими полями.

{\aim} данной работы является разработка микроскопической теории статических и динамических магнитоэлектрических эффектов в диэлектрике \cbo\ на основе квантово-механического подхода.

Для~достижения поставленной цели необходимо было решить следующие {\tasks}:
\begin{enumerate}[beginpenalty=10000] % https://tex.stackexchange.com/a/476052/104425
	\item проанализировать и систематизировать имеющиеся на момент написания работы данные по экспериментальному наблюдению и теоретическому описанию магнитоэлектрических эффектов в \cbo;
	\item получить уровни энергии и волновые функции иона меди 3d\(^9\) в кристалле \cbo
	\item оценить параметры взаимодействия электронов меди с электрическим и магнитным полем
	\item сопоставить результаты теоретических расчетов с имеющимися экспериментальными данными
\end{enumerate}


{\novelty}
\begin{enumerate}[beginpenalty=10000] % https://tex.stackexchange.com/a/476052/104425
	\item Как это подчеркивается в обзорах \autocite{Khomskii2009, Moskvin2009, Tokura2014, Shuai2015}, микроскопические механизмы магнитоэлектрической связи еще не вполне выяснены.
	\item Впервые построены теоретические диаграммы пространственной асимметрии люминесценции, которые не могут быть получены в рамках теоретико-группового подхода.
	\item Получен эффективный оператор статической магнитоэлектрической связи, отличный от используемых ранее для объяснения электрической поляризации в \ncbo.
\end{enumerate}


{\probation}
Основные результаты работы докладывались~на:
\begin{enumerate}[beginpenalty=10000] % https://tex.stackexchange.com/a/476052/104425
	\item \textit{Нурмухаметов, А. Р.} Особенности экситонных зон Френкеля-Давыдова в антиферромагнетиках [Устный доклад] / А.~Р.~Нурмухаметов // Итоговая конференция Института Физики. - 2021.
	\item \textit{Нурмухаметов, А. Р.} Магнитоэлектрическая связь в Cu\(_{1-x}\)Ni\(_{x}\)B\(_{2}\)O\(_{4}\) [Устный доклад] / А.~Р.~Нурмухаметов // Итоговая конференция Института Физики. - 2022.
	\item \textit{Нурмухаметов, А. Р.} К теории необратимости в спектрах \cbo\ [Стендовый доклад] / А.~Р.~Нурмухаметов, М.~В.~Еремин // Нанофизика и наноэлектроника. Труды XXVI Международного симпозиума. - 2022.
\end{enumerate}


